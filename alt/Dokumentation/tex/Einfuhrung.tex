\chapter{Vorwort}
Dieses Dokument beschreibt die Kenndaten und Hauptinhalte eines Projektes, das von mir Anfang 2011 begonnen wurde. Es sollen die Ideen und von mir genutzten Umsetzungsmöglichkeiten aufgezeigt und die aufgetretenen Probleme besprochen werden. In keinem Fall handelt es sich hierbei um eine komplette Einführung in die besprochenen Themengebiete oder eine nachvollziehbare Anleitung zum Nachbau des Datenhandschuhs. Vielmehr soll dieses Dokument die reine Möglichkeit und Umsetzbarkeit eines solchen Projektes aufzeigen.
\begin{flushright}
Bruchsal, \dates \\
Nils Braun 
\end{flushright}

\chapter{Abstrakt}
Der Datenhandschuh besteht aus mehreren Komponenten. Auf Hardwareseite stehen da vor allem die fünf Beschleunigungssensoren, von denen vier die Fingerkuppenbewegungen (außer dem kleinen Finger) aufnehmen und einer die ganze Handbewegung. Die analogen Spannungsdaten werden über einen AD-Wandler seriell an den Computer gesendet. Dort, durch die serielle Schnittstelle aufgenommen, gelangen die Daten zu einem vorher trainierten Neuronalen Netz und werden von diesem kategorisiert. Somit lassen sich einzelne Gesten voneinander unterscheiden und somit auf Benutzereingaben reagieren. Die ganze Bandbreite der möglichen Interaktion des Benutzers mit dem Computer - wie sie durch tastatur und Maus erreicht wird - lässt sich somit natürlich nicht abdecken. Jedoch bietet diese Art der Kommunikation eine Möglichkeit, auf ganz andere Aspekte der Ausdrucksformen einzugehen.
